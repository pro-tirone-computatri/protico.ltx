% This file is part of the Open Source project 'proTironeComputatri'
% (c) 2025 Karsten Reincke (https://github.com/pro-tirone-computatri/protico.ltx)
% It is distributed under the terms of the creative commons license
% CC-BY-4.0 (= https://creativecommons.org/licenses/by/4.0/)

\selectlanguage{ngerman}


\begin{frame}  
  \frametitle{LF09:03::Layer I-II: Kabeltypen }
  \begin{center}
    \includegraphics[height=7.5cm]{\imgGl/cable-types.png}
  \end{center}
\end{frame}


\begin{frame}  
  \frametitle{LF09:03::Layer I-II: Ethernetstandards }

  \begin{itemize}
    \item \textbf{Ethernet/10Base-T} = IEEE Standard 802.3 = 10 Mbps
    \item \textbf{Fast Ethernet/100Base-T} = IEEE Standard 802.3u = 100 Mbps
    \item \textbf{Gigabit Ethernet/GigE} = IEEE Standard 802.3z = 1000 Mbps
    \item \textbf{10 Gigabit Ethernet} = IEEE Standard 802.3ae = 10 Gbps
  \end{itemize}

\end{frame}

\begin{frame}  
  \frametitle{LF09:03::Layer I-II: Datendurchsatzberechnungen }

  \begin{center}
  \begin{footnotesize}

  \renewcommand{\arraystretch}{1.5}
  \begin{tabular}{|c|c|c|c|}
    \multicolumn{4}{c}{Präfixsystematik}  \\
    \hline
    Dec. & Size & Bin. & Size \\
    \hline
    \hline
    kilo & $10^3$ & kibi &  $2^{10}$ \\
    \hline
    mega & $10^6$ & mibi &  $2^{20}$  \\
    \hline
    giga & $10^9$ & gibi &  $2^{30}$  \\
    \hline
  \end{tabular}

  \vspace{0.4cm}
  \renewcommand{\arraystretch}{1.5}
  \begin{tabular}{|c|c|c|c|}
    \multicolumn{4}{c}{IT-Anwendung} \\
    \hline
    Dec. & Size & Bin. & Size\\
    \hline
    \hline
    1 Kilobyte &  $10^3$ Bytes & 1 Kibibyte = KiB & $2^{10}$ Bytes \\
    \hline
    1 Megabyte &  $10^6$ Bytes & 1 Mibibyte = MiB & $2^{20}$ Bytes \\
    \hline
    1 Gigabyte &  $10^9$ Bytes & 1 Gibibyte = GiB & $2^{30}$ Bytes \\
    \hline
  \end{tabular}

\end{footnotesize}
\end{center}

\vspace{0.4cm}
$\rightarrow$ Unterscheide auch \texttt{Mbps} von \texttt{Mibps} und \texttt{MB/s} von \texttt{MiB/s}!

\end{frame}

\begin{frame}  
  \frametitle{LF09:03::Layer I-II:Wlan-Standards nach Datendurchsatz }
  \begin{small}

  \begin{itemize}
    \item \textbf{IEEE 802.11a} = max 54 MBit/S im 5GHz Band
    \item \textbf{IEEE 802.11b} = max 11 MBit/S im 2,4 GHz Band
    \item \textbf{IEEE 802.11g} = max 54 MBit/S im 2,4 GHz Band
    \item \textbf{IEEE 802.11n} = Wi-Fi4 = erlaubt bis zu 289 MBit/s im 2,4 GHz Band und bis 600 MBit/s im 5GHz Band
    \item \textbf{IEEE 802.11ac} = Wi-Fi5 = seit 2014 mit max. 6933 Mbit/s nur 5GHz Ban
    \item \textbf{IEEE 802.11ax} = Wi-Fi6 = seit 2021 mit max. 9600 Mbit/s im 2,4-GHz- und im 5-GHz-Band arbeitet. 
    \item \textbf{IEEE 802.11be} = Wi-Fi7 = seit 2024 im 2,4-GHz-, 5-GHz- und 6-GHz-Band
    \item \textbf{IEEE 802.11 Mrd.} = Wi-Fi8 = voraussichtlich 2028 
  \end{itemize}
\end{small}

\end{frame}

\begin{frame}  
  \frametitle{LF09:03::Layer I-II:Wlan-Sicherheit }

  \subsubsection{Verschlüsselung}
  \begin{itemize}
    \item \textbf{WEP} = Wired Equivalent Privacy Verschlüsselung (RC4-Verschlüsselung)  
    \item \textbf{WPA} = Wi-Fi Protected Access (Wie WEP, aber mit TKIP (Temporal Key Integrity Protocol))
    \item \textbf{WPA2} = Wi-Fi Protected Access 2 
   \begin{itemize}
      \item verbesserte Verschlüsselungsmethode AES (Advanced Encryption Standard) 
      \item mit CCMP (Counter Mode with Cipher Block Chaining Message Authentication Code Protocol) 
    \end{itemize}
  \end{itemize}

  \subsubsection{Authentifizierung}
  \begin{itemize}
    \item \textbf{WPA-PSK}:  Kenntnis \textbf{Pre-Shared-Key} =  Nutzungsberechtigung.
    \item \textbf{Radius-Server} (Remote Authentication Dial-In User Service) : zentraler personenbezogener Authentifizierungsserver
  \end{itemize}

\end{frame}