% This file is part of the Open Source project 'proTironeComputatri'
% (c) 2025 Karsten Reincke (https://github.com/pro-tirone-computatri/protico.ltx)
% It is distributed under the terms of the creative commons license
% CC-BY-4.0 (= https://creativecommons.org/licenses/by/4.0/)

\begin{frame}[fragile]  
  \frametitle{LF09:08:Exkurs:Bitweise Verknüpfung}
  \begin{center}
    \begin{minipage}{0.9\textwidth}
      \inputminted[fontsize=\footnotesize,linenos]{python}{\snpLf/dm-0908-bitmask.py}
    \end{minipage}
  \end{center}
\end{frame}

\begin{frame}[fragile]  
  \frametitle{LF09:08:Exkurs:Bitweise Operatoren}
  \vspace{0.2cm}

  \begin{small}
  \textbf{Bitweise Operatoren} in Computersprachen
  \begin{itemize}
    \item sind \texttt{!} , \texttt{|} und \texttt{\&} 
    \item sind ein- oder zweistellige Funktionen \underline{in} der Menge $\{1,0\}$
    \item bilden `\{1,0\}` bzw. `\{1,0\}x\{1,0\}` auf `\{1,0\}` ab
  \end{itemize}
    
  \vspace{0.2cm}
  \begin{center}

      \renewcommand{\arraystretch}{1.3}
\begin{tabular}{c|c|c|c|}
  & \textit{not} = \texttt{!} & \textit{or} = \texttt{|} & \textit{and} = \texttt{\&} \\
  \hline
  R1 & \texttt{!1 $\rightarrow$ 0} & \texttt{1 | 1 $\rightarrow$ 1} & \texttt{1 \& 1 $\rightarrow$ 1} \\
  \hline
  R2 & \texttt{!0 $\rightarrow$ 1} & \texttt{1 | 0 $\rightarrow$ 1} & \texttt{1 \& 0 $\rightarrow$ 0} \\
  \hline
  R3 &  & \texttt{0 | 1 $\rightarrow$ 1} & \texttt{0 \& 1 $\rightarrow$ 0} \\
  \hline
  R4 &  & \texttt{0 | 0 $\rightarrow$ 0} & \texttt{0 \& 0 $\rightarrow$ 0} \\

\end{tabular}
    \end{center}
  \end{small}
  
  \vspace{0.5cm}
\begin{footnotesize}
    \begin{flushright}
      \textit{Bitweise Operatoren nehmen n Zahlen als Argumente und verknüpfen \\
      die darin gesetzten an derselben Position gesetzten Bits nach diesen Regeln}
    \end{flushright}
  \end{footnotesize}

\end{frame}

\begin{frame}[fragile]  
  \frametitle{LF09:08:Exkurs:Bitweise Operatoren}

    \begin{center}
    \begin{minipage}{0.9\textwidth}
  
      \inputminted[fontsize=\footnotesize,linenos]{python}{\snpLf/dm-0908-int-remains-int.py}
    \end{minipage}
    \end{center}


  \vspace{0.6cm}
  \begin{flushright}
    \begin{footnotesize}
      \textrm{\textbf{Integer bleibt Integer - unabhängig von der Darstellung!}}
      
      \vspace{0.3cm}
      \textrm{\textit{$\Rightarrow$ Vor bitweiser Verknüpfung}}\\
      \textrm{\textit{keine gesonderte Umwandlung}}\\
      \texttt{'dezimal $\rightarrow$ binär'} \textrm{\textit{nötig.}}\\
    \end{footnotesize}
  \end{flushright}
\end{frame}

\begin{frame}[fragile]  
  \frametitle{LF09:08:Exkurs:Logische Operatoren}
  \vspace{0.2cm}

  \begin{small}
  \textbf{Logische Operatoren} in Computersprachen
  \begin{itemize}
    \item sind \texttt{!} , \texttt{||} und \texttt{\&\&} 
    \item sind ein- oder zweistellige Funktionen \underline{in} der Menge $\{T,F\}$
    \item bilden `\{T,F\}` bzw. `\{T,F\}x\{T,F\}` auf `\{T,F\}` ab
  \end{itemize}
    
  \vspace{0.2cm}
  \begin{center}

      \renewcommand{\arraystretch}{1.3}
\begin{tabular}{c|c|c|c|}
  & \textit{NOT} = \texttt{!} & \textit{OR} = \texttt{||} & \textit{AND} = \texttt{\&\&} \\
  \hline
  R1 & \texttt{!T $\rightarrow$ F} & \texttt{T || T $\rightarrow$ T} & \texttt{T \&\& T $\rightarrow$ T} \\
  \hline
  R2 & \texttt{!F $\rightarrow$ T} & \texttt{T || F $\rightarrow$ T} & \texttt{T \&\& F $\rightarrow$ F} \\
  \hline
  R3 &  & \texttt{F || T $\rightarrow$ T} & \texttt{F \&\& T $\rightarrow$ F} \\
  \hline
  R4 &  & \texttt{F || F $\rightarrow$ F} & \texttt{F \&\& F $\rightarrow$ F} \\

\end{tabular}
    \end{center}
  \end{small}
  
  \vspace{0.5cm}
\begin{footnotesize}
    \begin{flushright}
      \textit{Logische Operatoren nehmen n Wahrheitswerte als Argumente \\
      und verknüpfen diese nach obigen Regeln}
    \end{flushright}
  \end{footnotesize}

\end{frame}

\begin{frame}[fragile]  
  \frametitle{LF09:08:Exkurs:Wahrheitswertetabellen}
  \vspace{0.2cm}

  \begin{small}
  \textbf{Logische Operatoren} in der Logik sind 
  \begin{itemize}
    \item sind \texttt{$\neg$} , \texttt{$\lor$} und \texttt{$\land$} 
    \item bilden `\{w,f\}` bzw. `\{w,f\}x\{w,f\}` auf `\{w,f\}` ab
  \end{itemize}
    
  \vspace{0.2cm}

Die Definition erfolgt durch Wahrheitswertetabellen:

  \begin{center}

      \renewcommand{\arraystretch}{1.3}
\begin{tabular}{c|c|c|c|c|c|}
p & q & $\neg$ p & $\neg$ q & p $\lor$ q & p $\land$ q \\
\hline
w & w & f & f & w & w \\
w & f & f & w & w & f \\
f & w & w & f & w & f \\
f & f & w & w & f & f 
\end{tabular}
    \end{center}
  \end{small}
  
  \vspace{0.5cm}
\begin{footnotesize}
    \begin{flushright}
      \textit{Mit Wahrheitswertetabellen kann man weitere \\
      'logische Operatoren' definieren, z.B. $\to$ (wenn, dann)}
    \end{flushright}
  \end{footnotesize}

\end{frame}

\begin{frame}[fragile]  
  \frametitle{LF09:08:Logische Operatoren:Was geht?}
  \begin{center}
    \begin{minipage}{0.6\textwidth}
      \inputminted[fontsize=\footnotesize,linenos]{python}{\snpLf/dm-0908-logop.py}
    \end{minipage}
      
    \vspace{1cm}
    \begin{minipage}{0.6\textwidth}
      \inputminted[fontsize=\footnotesize,linenos]{c++}{\snpLf/dm-0908-logop.cc}
    \end{minipage}
  \end{center}
\end{frame}

\begin{frame}[fragile]  
  \frametitle{LF09:08:Bitweise + logische Operatoren:Was geht?}
  \begin{center}
    \begin{minipage}{0.6\textwidth}
      \inputminted[fontsize=\footnotesize,linenos]{python}{\snpLf/dm-0908-log-riddle.py}
    \end{minipage}
  \end{center}
\end{frame}

