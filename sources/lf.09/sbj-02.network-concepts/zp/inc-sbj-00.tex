% This file is part of the Open Source project 'proTironeComputatri'
% (c) 2025 Karsten Reincke (https://github.com/pro-tirone-computatri/protico.ltx)
% It is distributed under the terms of the creative commons license
% CC-BY-4.0 (= https://creativecommons.org/licenses/by/4.0/)

\selectlanguage{ngerman}


\begin{frame}  
  \frametitle{LF09:02:Ein Wust von Begriffen ...}
  \begin{center}
    \includegraphics[height=7.5cm]{\imgLf/0902-nwcs.png}
  \end{center}
\end{frame}

\begin{frame}
  \frametitle{LF09:02:1A ... geordnet anhand des OSI-Models}
	\begin{itemize}
		\item \textit{Layer VII}: \textbf{Application Layer} (= Anwendungsschicht)
		\item \textit{Layer VI}: \textbf{Presentation Layer} (= Darstellungsschicht)
		\item \textit{Layer V}: \textbf{Session Layer} (= Kommunikationsschicht)  
		\item \textit{Layer IV}: \textbf{Transport Layer} (= Transportschicht) 
		\item \textit{Layer III}: \textbf{Network Layer} (= Vermittlungsschicht) 
		\item \textit{Layer II}: \textbf{Data Link Layer} (= Sicherungsschicht) 
		\item \textit{Layer I}: \textbf{Physical Layer} (= Physikalische Schicht) 
	\end{itemize}
\end{frame}

\begin{frame}
  \frametitle{LF09:02:... geordnet anhand des OSI-Models}
	\begin{itemize}
		\item \textit{Layer VII}: \textbf{Application Layer} $\rightarrow$ Client-Server-Lösungen
		\item \textit{Layer VI}: \textbf{Presentation Layer} $\rightarrow$ MPEG, PNG, GIF, ASCII, UTF8, ....
		\item \textit{Layer V}: \textbf{Session Layer} $\rightarrow$ http-Protokoll, smtp-Protokoll, ssh 
		\item \textit{Layer IV}: \textbf{Transport Layer} $\rightarrow$ TCP-Protokoll, UDP-Protokoll, ...
		\item \textit{Layer III}: \textbf{Network Layer} $\rightarrow$ Routing, IP-Protokoll \& -Adressen, Ports
		\item \textit{Layer II}: \textbf{Data Link Layer} $\rightarrow$ Switch, Hardwareadressen, MAC-Adresse
		\item \textit{Layer I}: \textbf{Physical Layer} $\rightarrow$ (( [L$|$W]AN: Kupferkabel $|$ Glasfaser )  $|$ [WLAN]: Richtfunk $|$ Satelliten-Funk ), Signalformen, Frequenzen, TRANSmitter+reCIEVER, RJ-45 Kabel, HUBS, CSMA/CD, Token-Ring, Kollisionsvermeidung
	\end{itemize}
\end{frame}

\begin{frame}
  \frametitle{LF09:02:... oder des TCP-Interoperability-Models}

  \begin{footnotesize}
    \begin{tabularx}{\textwidth}{|c|c|c|X|}
      * & OSI-Schichten & TCP/IP-Schichten & Beispiele \\
      \hline
      \hline
      \textit{VII} & \textbf{Anwendungen} & \textbf{Anwendungen} & HTTP, UDS, FTP, SMTP, POP, Telnet, DHCP, OPC UA \\
      \textit{VI} & \textbf{Darstellung} & & \\
      \textit{V} & \textbf{Sitzung} & & TLS, SOCKS \\
      \hline
      \textit{IV} & \textbf{Transport} & \textbf{Transport} & TCP, UDP, SCTP \\
      \hline
      \textit{III} & \textbf{Vermittlung} & \textbf{Internet} & IP (IPv4, IPv6), ICMP (über IP) \\
      \hline
      \textit{II} & \textbf{Sicherung} & & Ethernet, Token Bus, Token Ring, FDDI \\
      \textit{I} & \textbf{Bit-Übertragung} & \textbf{Netzzugang} & \\
      \hline
      \hline
  
    \end{tabularx}
  \end{footnotesize}
\end{frame}

\begin{frame}  
  \frametitle{LF09:02:Netzwerktopologien}
  \begin{center}
    \includegraphics[height=5cm]{\imgLf/network-topologies-wm.png}
  \end{center}
\end{frame}

\begin{frame}  
  \frametitle{LF09:02:Netzwerkreichweiten}
  \begin{center}
    \includegraphics[height=7.5cm]{\imgLf/network-area-taxonomy.png}
  \end{center}
\end{frame}

\begin{frame}  
  \frametitle{LF09:02:Netzwerkdesign als Projektkommunikation}
  \begin{center}
    \includegraphics[width=12cm]{\imgLf/network-target-example.png}
  \end{center}
\end{frame}
