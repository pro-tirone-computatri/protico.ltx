% This file is part of the Open Source project 'proTironeComputatri'
% (c) 2025 Karsten Reincke (https://github.com/pro-tirone-computatri/protico.ltx)
% It is distributed under the terms of the creative commons license
% CC-BY-4.0 (= https://creativecommons.org/licenses/by/4.0/)

\selectlanguage{ngerman}

\begin{frame}[fragile]  
  \frametitle{LF09:11:IPv6-Adressen}
  \vspace{0.2cm}
  \begin{tiny}
  \begin{center}
    \renewcommand{\arraystretch}{2.4}
    \begin{tabular}{|c|c|c|c|c|c|c|c|c|c|c|c|c|c|c|c|}
    \hline
    \textcolor{gray}{\tiny{F.}} & \textcolor{gray}{\tiny{E.}} &
        \textcolor{gray}{\tiny{D.}} & \textcolor{gray}{\tiny{C.}} &
          \textcolor{gray}{\tiny{B.}} & \textcolor{gray}{\tiny{A.}} &
            \textcolor{gray}{\tiny{9.}} & \textcolor{gray}{\tiny{8.}} &
              \textcolor{gray}{\tiny{7.}} & \textcolor{gray}{\tiny{6.}} &
                \textcolor{gray}{\tiny{5.}} & \textcolor{gray}{\tiny{4.}} &
                  \textcolor{gray}{\tiny{3.}} & \textcolor{gray}{\tiny{2.}} &
                   \textcolor{gray}{\tiny{1.}} & \textcolor{gray}{\tiny{0.}} \\
    \hline
    \textcolor{blue}{\tiny{20}} & \textcolor{blue}{\tiny{01}} &
        \textcolor{blue}{\tiny{0d}} & \textcolor{blue}{\tiny{b8}} &
          \textcolor{blue}{\tiny{00}} & \textcolor{blue}{\tiny{00}} &
            \textcolor{blue}{\tiny{08}} & \textcolor{blue}{\tiny{d3}} &
              \textcolor{blue}{\tiny{00}} & \textcolor{blue}{\tiny{00}} &
                \textcolor{blue}{\tiny{8a}} & \textcolor{blue}{\tiny{2e}} &
                  \textcolor{blue}{\tiny{00}} & \textcolor{blue}{\tiny{70}} &
                   \textcolor{blue}{\tiny{73}} & \textcolor{blue}{\tiny{44}} \\
    \hline
    \multicolumn{2}{|c|}{2001} & \multicolumn{2}{c|}{0db8} &
      \multicolumn{2}{c|}{0000} & \multicolumn{2}{c|}{08d3} &
        \multicolumn{2}{c|}{0000} & \multicolumn{2}{c|}{8a2e} &
          \multicolumn{2}{c|}{0070} & \multicolumn{2}{c|}{7344} \\
    \hline
    \multicolumn{16}{|c|}{
      \textcolor{teal}{\footnotesize{\textbf{2001:\textcolor{magenta}{0}db8:\textcolor{magenta}{000}0:\textcolor{magenta}{0}8d3:\textcolor{magenta}{000}0:8a2e:\textcolor{magenta}{00}70:7344}}}} \\
    \hline
    \hline
    \multicolumn{16}{|c|}{\textcolor{gray}{führende Nullen im Bytepaar dürfen weggelassen werden}} \\
    \hline
    \multicolumn{2}{|c|}{2001} & \multicolumn{2}{c|}{db8} &
      \multicolumn{2}{c|}{0} & \multicolumn{2}{c|}{8d3} &
        \multicolumn{2}{c|}{0} & \multicolumn{2}{c|}{8a2e} &
          \multicolumn{2}{c|}{70} & \multicolumn{2}{c|}{7344} \\
    \hline
    \multicolumn{16}{|c|}{\textcolor{teal}{\small{\textbf{2001:db8:\textcolor{orange}{0}:8d3:\textcolor{orange}{0}:8a2e:70:7344}}}} \\
    \hline
    \hline
    \multicolumn{16}{|c|}{\textcolor{gray}{1-mal dürfen alle 0er Bytepaare hintereinander weggelassen werden}} \\
    \hline
    \multicolumn{2}{|c|}{2001} & \multicolumn{2}{c|}{db8} &
      \multicolumn{2}{c|}{} & \multicolumn{2}{c|}{8d3} &
        \multicolumn{2}{c|}{0} & \multicolumn{2}{c|}{8a2e} &
          \multicolumn{2}{c|}{70} & \multicolumn{2}{c|}{7344} \\
    \hline
    \multicolumn{16}{|c|}{\textcolor{teal}{\normalsize{\textbf{2001:db8::8d3:\textcolor{orange}{0}:8a2e:70:7344}}}} \\
   
    \hline
    \hline
  \end{tabular}
\end{center}
\end{tiny}
\end{frame}

\begin{frame}[fragile]  
  \frametitle{LF09:11:IPv6-Adressen:Anzahl}

  \vspace{0.2cm}
  \begin{tiny}
    
  \begin{center}
    \renewcommand{\arraystretch}{3}
    \begin{tabular}{|c|c|c|c|c|c|c|c|c|c|c|c|c|c|c|c|}
      \hline
      \textcolor{gray}{\tiny{F.}} & \textcolor{gray}{\tiny{E.}} &
        \textcolor{gray}{\tiny{D.}} & \textcolor{gray}{\tiny{C.}} &
          \textcolor{gray}{\tiny{B.}} & \textcolor{gray}{\tiny{A.}} &
            \textcolor{gray}{\tiny{9.}} & \textcolor{gray}{\tiny{8.}} &
              \textcolor{gray}{\tiny{7.}} & \textcolor{gray}{\tiny{6.}} &
                \textcolor{gray}{\tiny{5.}} & \textcolor{gray}{\tiny{4.}} &
                  \textcolor{gray}{\tiny{3.}} & \textcolor{gray}{\tiny{2.}} &
                   \textcolor{gray}{\tiny{1.}} & \textcolor{gray}{\tiny{0.}} \\
      \hline
      \textcolor{blue}{\tiny{$2^8$}} & \textcolor{blue}{\tiny{$2^8$}} &
        \textcolor{blue}{\tiny{$2^8$}} & \textcolor{blue}{\tiny{$2^8$}} &
          \textcolor{blue}{\tiny{$2^8$}} & \textcolor{blue}{\tiny{$2^8$}} &
            \textcolor{blue}{\tiny{$2^8$}} & \textcolor{blue}{\tiny{$2^8$}} &
              \textcolor{blue}{\tiny{$2^8$}} & \textcolor{blue}{\tiny{$2^8$}} &
                \textcolor{blue}{\tiny{$2^8$}} & \textcolor{blue}{\tiny{$2^8$}} &
                  \textcolor{blue}{\tiny{$2^8$}} & \textcolor{blue}{\tiny{$2^8$}} &
                   \textcolor{blue}{\tiny{$2^8$}} & \textcolor{blue}{\tiny{$2^8$}} \\
      \hline
      \multicolumn{16}{|c|}{\footnotesize{\textcolor{blue}{
        \texttt{$2^{(8 + 8 + 8 + 8 + 8 + 8 + 8 + 8 + 8 + 8 + 8 + 8 + 8 + 8 + 8 + 8 + 8)}$}}}} \\
    \hline
    \multicolumn{16}{|c|}{\small{\textcolor{blue}{\texttt{$= 2^{(128)}$}}}} \\
    \hline
    \multicolumn{16}{|c|}{\normalsize{\textcolor{teal}{\texttt{$= 3,402823669*10^{38}$}}}} \\
    \hline
    \multicolumn{16}{|c|}{\normalsize{\textcolor{magenta}{\texttt{ ~ $30$ Sextillionen}}}} \\
    \hline
  \end{tabular}
\end{center}
\end{tiny}

\vspace{0.4cm}
\begin{flushright}
\begin{tiny}
$2^{128} / 8.000.000.000\ Menschen = 3,402823669*10^{38} / (8*10^{9})  = 4.253529586*10^{28}\ pro\ Mensch$ = unvorstellbar groß
\end{tiny}
\end{flushright}

\end{frame}


\begin{frame}
  \frametitle{LF09:11:IPv6-Adresstypen}

  \begin{footnotesize}

  \begin{itemize}
    \item \textbf{Unspecified Address} = \texttt{0:0:0:0:0:0:0:0/128} = \texttt{::/128} \\
    \begin{tiny}{\textit{ Entspricht der IPv4-Adresse 0.0.0.0/32}}\end{tiny}

    \item \textbf{Global Unicast Address} =  weltweit gültig, ins Internet routbar \\
    \begin{tiny}{Erkennungspräfix: \textcolor{magenta}{\texttt{2000::/3}} 
      ( \texttt{0x[20|00] = \textcolor{teal}{001}00000|00000000} })
    \end{tiny}

    \item \textbf{Link Local Unicast-Address} = nur in BCD gültig, nicht routbar, wie MAC-Adresse\\
    \begin{tiny}{Erkennungspräfix: \textcolor{magenta}{\texttt{fe80::/10}} 
      ( \texttt{0x[fe|80] = \textcolor{teal}{111111110}|\textcolor{teal}{10}0000000})}
    \end{tiny}

    \item \textbf{Unique Local Address} = ULA, nicht ins Internet routbar, wie private IPv4-Adressen \\
    \begin{tiny}{Erkennungspräfix: \textcolor{magenta}{\texttt{fc00::/7}} 
      ( \texttt{0x[fc|00] = \textcolor{teal}{1111110}0|000000000})}\end{tiny}
    
    \item \textbf{Loopback Address} = 'Selbstgespräch'-Adresse, wie superprivat IPv4-Adressen \\
    \begin{tiny}Erkennungspräfix: \textcolor{magenta}{\texttt{::1/128}} \end{tiny}


  \end{itemize}

  \begin{tiny}

  \begin{itemize}
    \item \textbf{Unicast Address} :- One-To-One-Adresse = 1 Adresse, 1 Interface

    \item \textbf{Multicast Address} = One-to-Many-Adresse = 1 Adresse aufgelöst auf viele Interfaces\\
    \begin{tiny}{Erkennungspräfix: \textcolor{magenta}{\texttt{ff00::/8}} 
      ( \texttt{0x[ff|00] = \textcolor{teal}{111111111}|000000000})}
    \end{tiny}

    \item \textbf{Anycast Address} = One-to-Nearest-Adresse = 1 Adresse aufgelöst auf die am besten passende\\
    = netzwerktechnisch 'nächste'. Erkennungspräfix: Wie Global Unicast Adressen. Unterschied konfigurativ.

  \end{itemize}
  \end{tiny}

  \end{footnotesize}
\end{frame}

\begin{frame}[fragile]  
  \frametitle{LF09:11:IPv6-Segmentierung für Global Unicast Addr.}
  \vspace{0.2cm}
  \begin{tiny}

  \begin{itemize}
    \item \textcolor{red}{IANA} $\rightarrow$ \textcolor{orange}{RIR} (Regional Internet Registry): /32-Ipv6-Netz
    \item \textcolor{orange}{RIR} $\rightarrow$ \textcolor{blue}{ISP} (Internetprovider): /48-IPv6-Netz
    \item \textcolor{blue}{ISP} $\rightarrow$ \textcolor{teal}{Kunde}: /56-IPv6-Netz
    \item \textcolor{teal}{Kunde} $\rightarrow$ \textcolor{gray}{Gerät} : 64Bit-Identifier
  \end{itemize}


  \begin{center}
    \renewcommand{\arraystretch}{3}
    \begin{tabular}{|c|c|c|c|c|c|c|c|c|c|c|c|c|c|c|c|}
    \hline
     \multicolumn{8}{|c|}{\footnotesize{\textcolor{magenta}{Netzpräfix (64 Bit)}}} &
     \multicolumn{8}{c|}{\footnotesize{\textcolor{gray}{Interface Identifier (64 Bit)}}}\\
    \hline
    \textcolor{gray}{\tiny{F.}} & \textcolor{gray}{\tiny{E.}} &
        \textcolor{gray}{\tiny{D.}} & \textcolor{gray}{\tiny{C.}} &
          \textcolor{gray}{\tiny{B.}} & \textcolor{gray}{\tiny{A.}} &
            \textcolor{gray}{\tiny{9.}} & \textcolor{gray}{\tiny{8.}} &
              \textcolor{gray}{\tiny{7.}} & \textcolor{gray}{\tiny{6.}} &
                \textcolor{gray}{\tiny{5.}} & \textcolor{gray}{\tiny{4.}} &
                  \textcolor{gray}{\tiny{3.}} & \textcolor{gray}{\tiny{2.}} &
                   \textcolor{gray}{\tiny{1.}} & \textcolor{gray}{\tiny{0.}} \\

    \hline
    \textcolor{red}{\tiny{20}} & \textcolor{red}{\tiny{01}} &
        \textcolor{red}{\tiny{0d}} & \textcolor{red}{\tiny{b8}} &
          \textcolor{orange}{\tiny{85}} & \textcolor{orange}{\tiny{a3}} &
            \textcolor{blue}{\tiny{08}} & \textcolor{teal}{\tiny{d3}} &
              \textcolor{gray}{\tiny{00}} & \textcolor{gray}{\tiny{00}} &
                \textcolor{gray}{\tiny{8a}} & \textcolor{gray}{\tiny{2e}} &
                  \textcolor{gray}{\tiny{00}} & \textcolor{gray}{\tiny{70}} &
                   \textcolor{gray}{\tiny{73}} & \textcolor{gray}{\tiny{44}} \\
    \hline
     \multicolumn{4}{|c|}{\footnotesize{\textcolor{red}{IANA}}} &
     \multicolumn{2}{c|}{\footnotesize{\textcolor{orange}{RIR}}} &
     \multicolumn{1}{c|}{\footnotesize{\textcolor{blue}{ISP}}} &
     \multicolumn{1}{c|}{\footnotesize{\textcolor{teal}{Kde}}} &
     \multicolumn{8}{c|}{\footnotesize{\textcolor{gray}{Interface Identifier (64 Bit)}}}\\   
     
     
    \hline

    \multicolumn{2}{|c|}{2001} & \multicolumn{2}{c|}{0db8} &
      \multicolumn{2}{c|}{85a3} & \multicolumn{2}{c|}{08d3} &
        \multicolumn{2}{c|}{0000} & \multicolumn{2}{c|}{8a2e} &
          \multicolumn{2}{c|}{0070} & \multicolumn{2}{c|}{7344} \\
    \hline
    \multicolumn{6}{|c|}{\tiny{\textcolor{red}{Routing-Präfix}}} &
    \multicolumn{2}{c|}{\tiny{\textcolor{blue}{Subnet-Ids}}} &
    \multicolumn{8}{c|}{\footnotesize{\textcolor{gray}{Interface Identifier (64 Bit)}}}\\  
    \hline
    \end{tabular}
  \end{center}
\end{tiny}

\vspace{0.4cm}
\begin{flushright}
\begin{tiny}
nach  \href{https://de.wikipedia.org/wiki/IPv6}{https://de.wikipedia.org/wiki/IPv6}
\end{tiny}
\end{flushright}
\end{frame}

\begin{frame}
  \frametitle{LF09:11:Ergänzende Hinweise}
  \begin{footnotesize}
    \begin{itemize}
      \item Interface Identifier hat immer 64 Bit.
      \item Demselben Interface können mehrere IPv6-Adressen zugeordnet sein.
      \item Die Interface-ID kann folglich in mehreren IPv6-Adressen auftauchen.
      \item Eine \textbf{Global Unicast Address} beginnt mit einem Routing-Präfix (48 Bits = 6 Bytes), 
      \item Im Routing-Präfix sind 'nur' 13 Bits für die Top-Level-Aggregation vorgesehen.   
      \item 13 Bits = 8191 Toplevel Provider = RIRs.
      \item In Routingtabellen werden nur aggregierte Netze verwaltet.
    \end{itemize}
  \end{footnotesize}
  \vspace{0.4cm}
  \begin{tiny}
      \cite[vgl. dazu][188ff]{Schreiner2014a}
  \end{tiny}
\end{frame}

\begin{frame}
  \frametitle{LF09:11:IPv6-Segmentierung:Übung}
  \begin{center}
    \includegraphics[height=7.2cm]{\imgLf/ipv6-segmentation-task.png}
  \end{center}
\end{frame}

\begin{frame}
  \frametitle{LF09:11:IPv6-Segmentierung:Lösung}
  \begin{center}
    \includegraphics[height=7.5cm]{\imgLf/ipv6-segmentation-solution.png}
  \end{center}
\end{frame}

\begin{frame}
  \frametitle{LF09:11: die MAC-Adresse als Link Local Unicast Addr.}

  \begin{footnotesize}

  \begin{center}
 
 \begin{bfseries}Link Local Unicast Address\end{bfseries} = \textcolor{red}{nicht routbare Adresse im lokalen Netz}

 \vspace{0.5cm}
  $\Rightarrow$

 \vspace{0.5cm}
  sozusagen eine  \begin{bfseries}\textcolor{blue}{IPv6-Mac-Adresse}\end{bfseries}

 \vspace{0.5cm}
  $\Rightarrow$

 \vspace{0.5cm}
  Warum die nicht aus realer MAC-Adresse 'konstruieren'?
 
  \vspace{0.5cm}

  \end{center}
  \begin{tiny}
    \textcolor{blue}{Idee:} \textcolor{gray}Link Local Unicast Prefix \texttt{FE80} + viele 0en + 6Byte Mac-Adresse?
  \end{tiny}
\end{footnotesize}

\end{frame}



\begin{frame}
  \frametitle{LF09:11: die MAC-Adresse als Link Local Unicast Addr.}

  \begin{footnotesize}

  \begin{itemize}
    \item Mac-Adresse (Hex-Notation) in zwei 3-Byte-Blöcke aufteilen:\\
    \qquad \texttt{52:74:f2:b1:a8:7f} $\rightarrow$ \texttt{52:74:f2} und \texttt{b1:a8:7f}.
    \item Dazwischen den Wert \texttt{FF:FE} einfügen: \\
    \qquad \texttt{52:74:f2} + \texttt{FF:FE} + \texttt{b1:a8:7f} $\rightarrow$ \texttt{52:74:f2:FF:FE:b1:a8:7f}.
    \item Das Ergebnis in die IPv6-Format umwandeln: \\
    \qquad \texttt{52:74:f2:FF:FE:b1:a8:7f} $\rightarrow$ \texttt{5274:f2ff:feb1:a87f}.
    \item Das erste Oktett in Binärnotation umwandeln: \\
    \qquad \texttt{52} $\rightarrow$ \texttt{01010010}
    \item Das 7 Bit invertieren: \\
    \qquad \texttt{01010010} $\rightarrow$ \texttt{01010000}.
    \item Ergebnis in Hex-Notation umwandeln:\\ 
    \qquad \texttt{01010000} $\rightarrow$ \texttt{50}.
    \item Das erste Oktett durch den neuen Wert ersetzen \\
    \qquad \texttt{5274:f2ff:feb1:a87f} $\rightarrow$ \texttt{5074:f2ff:feb1:a87f}.
    \item Dem Ganzen das Link-Local-Prefix voransetzen: \\
    \qquad \texttt{fe80::} voran ( \texttt{5074:f2ff:feb1:a87f} $\rightarrow$ \texttt{fe80::5074:f2ff:feb1:a87f}.
  \end{itemize}


\end{footnotesize}

\end{frame}


\begin{frame}
  \frametitle{LF09:11:IPv6-Vorteile}

  \begin{footnotesize}

  \begin{itemize}
    \item Vergrößerung des Adressraums. 
    \item Vereinfachung der Protokolle, weniger Rechenaufwand für Router. 
    \item DHCP vielfach überflüssig.
    \item Implementierung von IPsec innerhalb des IPv6-Standards.
    \item Immanente Unterstützung von Netztechniken wie Quality of Service und Multicast.
  \end{itemize}

\end{footnotesize}

\end{frame}


