% This file is part of the Open Source project 'proTironeComputatri'
% (c) 2025 Karsten Reincke (https://github.com/pro-tirone-computatri/protico.ltx)
% It is distributed under the terms of the creative commons license
% CC-BY-4.0 (= https://creativecommons.org/licenses/by/4.0/)

\selectlanguage{ngerman}


\begin{frame}  
  \frametitle{LF09:01:Fadennetzwerk}
  \begin{center}
    \includegraphics[height=5cm]{\imgLf/tangle-592987-pxh.png}
  \end{center}

  \begin{flushright}
    \textcolor{blue}{Mit \textit{Schere, Faden, Köpfchen und sich selbst}}
  \end{flushright}
\end{frame}

\begin{frame}  
  \frametitle{LF09:01 Fadennetzwerk / Aufgabe}
  \begin{columns}
  \column{0.6\textwidth}
  \begin{itemize}
    \item Netzwerk bauen aus/mit Bindfäden, Scheren und sich selbst
    \item Darin frei wählbare, einstellige Zahl von beliebigem Endgerät zu beliebigem Endgerät übertragen können.
    \item Bei der Vorführung nicht sprechen, nicht lächeln, nicht blinzeln oder sonst wie kommunizieren. 
    \item Maximale Übertragungslänge von ca. 2 Metern pro Faden.
    \item Zahl über mindestens ZWEI Fäden maximaler Länge übertragen
  \end{itemize}
  \column{0.4\textwidth}
  \begin{center}
    \includegraphics[width=4cm]{\imgLf/tangle-592987-pxh.png}
  \end{center}
\end{columns}
\end{frame}



\begin{frame}  
  \frametitle{LF09:01:Fadennetzwerk:Sterntopologie}
  \begin{center}
    \includegraphics[width=10cm]{\imgLf/fnw-stern-tpl.png}
  \end{center}
\end{frame}

\begin{frame}  
  \frametitle{LF09:01:Fadennetzwerk:Ringtopologie}
  \begin{center}
    \includegraphics[width=10cm]{\imgLf/fnw-ring-tpl.png}
  \end{center}
\end{frame}

\begin{frame}  
  \frametitle{LF09:01:Fadennetzwerk:Package Hopping}
  \begin{center}
    \includegraphics[width=12cm]{\imgLf/package-hopping.png}
  \end{center}
\end{frame}