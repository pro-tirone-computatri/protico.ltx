% This file is part of the Open Source project 'proTironeComputatri'
% (c) 2025 Karsten Reincke (https://github.com/pro-tirone-computatri/protico.ltx)
% It is distributed under the terms of the creative commons license
% CC-BY-4.0 (= https://creativecommons.org/licenses/by/4.0/)

\begin{frame}
  \frametitle{LF09:10:Traditionelle IPv4-Adressklassen}

\begin{footnotesize}

  \begin{tabular}{|c|r|c|}
 
  \hline
  \multirow{3}{*}{A} & \textbf{0.0.0.0} (= \texttt{0x00000000}) - \textbf{127.255.255.255} (= \texttt{0x7FFFFFFF}) & \tiny{\textcolor{red}{0b0}[01]\{31\}}\\
    \cline{2-3} 
    & \tiny{\textit{0.0.0.0} (= \texttt{0x00000000})}  [\tiny{kontextsensitiver Platzhalter}]  & \tiny{reserviert}\\
    \cline{2-3} 
    & \tiny{\textit{10.0.0.0} (= \texttt{0x0A000000})} - \tiny{\textit{10.255.255.255} (= \texttt{0x0AFFFFFF})} & \tiny{privat}\\
    \cline{2-3} 
    & \tiny{\textit{127.0.0.0} (= \texttt{0x7F000000})} - \tiny{\textit{127.255.255.255} (= \texttt{0x7FFFFFFF})} & \tiny{superprivat} \\
  \hline
  \multirow{3}{*}{B} & \textbf{128.0.0.0} (= \texttt{0x80000000}) - \textbf{191.255.255.255} (= \texttt{0xBFFFFFFF}) & \tiny{\textcolor{red}{0b10}[01]\{30\}} \\
    \cline{2-3} 
    & \tiny{\textit{169.254.0.1}} (= \texttt{0xA9FE0001}) - \tiny{\textit{169.254.255.255} (= \texttt{0xA9FEFFFF})} & \tiny{APIPA}\\
    \cline{2-3} 
    & \tiny{\textit{172.16.0.0} (= \texttt{0xAC100000})} - \tiny{\textit{172.31.255.255} (= \texttt{0xAC1FFFFF})} & \tiny{privat}\\
  \hline
  \multirow{3}{*}{C} & \textbf{192.0.0.0} (= \texttt{0xC0000000}) - \textbf{223.255.255.255} (= \texttt{0xDFFFFFFF}) & \tiny{\textcolor{red}{0b110}[01]\{29\}} \\
    \cline{2-3} 
    & \tiny{\textit{192.168.0.0}} (= \texttt{0xC0A80000}) - \tiny{\textit{192.168.255.255}} (= \texttt{0xC0A8FFFF}) & \tiny{privat}\\
  \hline
  D & 224.0.0.0 (= \texttt{0xE0000000}) - \texttt{239.255.255.255} (= \texttt{0xEFFFFFFF}) & \tiny{Multicast} \\
  \hline
  E & 240.0.0.0 (= \texttt{0xF0000000}) - \texttt{255.255.255.255} (= \texttt{0xFFFFFFFF}) & \tiny{Test}\\
  \hline
\end{tabular}

\end{footnotesize}

\end{frame}

\begin{frame}
  \frametitle{LF09:10:Exkurs:Regular Expressions}

  \begin{center}
  \textcolor{red}{0b0}[01]\{31\} = \\ alle Bitstrings beginnend mit \texttt{0b0}, gefolgt von 31 `0` oder `1`
  \end{center}
  \vspace{0.6cm}

  \begin{footnotesize}
    Notiert sein kann das Zeichen an einer Position - z.B. im String \texttt{AbbA}:
  
    \vspace{0.2cm}
  

    \begin{enumerate}
      \item \textbf{literal}: \textcolor{blue}{AbbA}
      \item \textbf{als Variante}: \textcolor{blue}{[Aa][Bb][Bb][Aa]} ($\Rightarrow$ \textcolor{gray}{aBBa} | \textcolor{gray}{AbbA} | ...)
      \item \textbf{quantifiziert}: \textcolor{blue}{Ab\{1,2\}A} ($\Rightarrow$ \textcolor{gray}{AbA}) | \textcolor{gray}{AbbA})
      \item \textbf{als quantifizierte Variante}: \textcolor{blue}{[[Aa][Bb]]\{2\}} ($\Rightarrow$ \textcolor{gray}{aBBa} | \textcolor{gray}{AbbA} | ...)
      \item  \textbf{mit Spezial-Quantoren}: 
      \begin{itemize}
        \item \textcolor{blue}{Abb\textcolor{red}{?}A} ($\Rightarrow$ \textcolor{gray}{AbA} | \textcolor{gray}{AbbA})
        \item \textcolor{blue}{Ab\textcolor{red}{+}A} ($\Rightarrow$ \textcolor{gray}{AbA} | ... | \textcolor{gray}{AbbbbbbA} | ...)
        \item \textcolor{blue}{Ab\textcolor{red}{*}A} ($\Rightarrow$ \textcolor{gray}{AA} | ... | \textcolor{gray}{AbbbbbbA} | ...)
      \end{itemize}
    \end{enumerate}
  \end{footnotesize}
\end{frame}

\begin{frame}
  \frametitle{LF09:10:Traditionelle IPv4-Adressklassen}
  \begin{center}
     \includegraphics[height=7cm]{\imgGl/ipv4-classes.png}
  \end{center}

\end{frame}

\begin{frame}
  \frametitle{LF09:10:Netzsegmentierung}
  \begin{center}
     \includegraphics[height=7cm]{\imgGl/split-680866-pxh.png}
  \end{center}

\end{frame}

\begin{frame}
  \frametitle{LF09:10:Netzsegmentierung als Abspaltung}
  \begin{center}
     \includegraphics[height=6.8cm]{\imgGl/ipv4-segmentation-splitting.png}
  \end{center}

  \begin{flushright}
  \tiny{mit Dank an Dennis, 11IP24, GS-LDK}
  \end{flushright}
\end{frame}


\begin{frame}

  \frametitle{LF09:10:IPv4-Segmentierung (mit Gateway-Beziehung)}
  \begin{columns}
    \column{0.75\textwidth}
      \begin{center}
        \includegraphics[height=7.8cm]{\imgGl/ipv4-segmentation.png}
      \end{center}
    \column{0.25\textwidth}
      \begin{tiny}1.) IPv4-Übermittlung unabhängig von IPv4-Adresssystematik.

\vspace{0.3cm}      
2.) Nachrichtenübermittlung mittels Hopping in benachbarte Broadcastdomainen.

\vspace{0.3cm} 
3.) Hopping = Wechsel andere Netz; Netzklassifikation dabei/dafür unerheblich.
      
\vspace{0.3cm} 
4.) Router = in mehrere Netz eingebunden

\vspace{0.3cm} 
5.) Paketumschreibungen auf Layer-II/III-Ebene durch Router

      \end{tiny}
  \end{columns}  
\end{frame}

\begin{frame}

  \frametitle{LF09:10:IPv4-Segmentierung als Prozess}
  \begin{center}
     \includegraphics[width=0.9\textwidth]{\imgLf/ipv4-netdesign-process.png}
  \end{center}
  
  \begin{footnotesize}
    \begin{itemize}
      \item \textcolor{teal}{\textit{Heuristik A}}: Rechner, die viel  miteinander kommunizieren, gehören in dieselbe Broadcastdomäne!\footnote{\tiny{Denn jedes Hopping verlangsamt die Kommunikation, weil dabei zusätzlich Layer-3-Berechnungen nötig werden.}}
      \item \textcolor{teal}{\textit{Heuristik B}}: Rechner, denen der Zugriff auf eine andere Broadcastdomäne mittels Firewallregeln erlaubt bzw. verboten werden soll, gehören in dieselbe Broadcastdomäne!\footnote{\tiny{Denn aufgesplittete Firewallregeln, die auf einzelne Rechner zugreifen, verlangsamen die Zugriffsberechnung. Und sie erschweren Netzwerkupdates}}
    \end{itemize}
\end{footnotesize}
\end{frame}

\begin{frame}[fragile]  
  \frametitle{LF09:10:Lösung-A:IPv4-Segmentierung}
  \begin{center}
    \includegraphics[height=7.4cm]{\imgLf/ipv4-segmentation-exercise.png}
  \end{center}
\end{frame}


\begin{frame}[fragile]  
  \frametitle{LF09:10:Uebung-B:IPv4-Segmentierung}
  \begin{center}
    \includegraphics[height=7.8cm]{\imgLf/ipv4-segmentation-task.png}
  \end{center}
\end{frame}

\begin{frame}[fragile]  
  \frametitle{LF09:10::Lösung-B:Initiale Bedarfsanalyse}

  \vspace{1cm}
  \begin{tiny}
  \renewcommand{\arraystretch}{1.5}
  \begin{tabular}{|c|c|c|c|c|c|c||c||c|c|c|c|}
    
    & \multicolumn{4}{c|}{Employees} %2-4 
    & \multirow{2}{*}{$S$} %5-6
    & \multirow{2}{*}{$\frac{C}{E}$} %7
    & $\sum_{25}^{27}{}*\frac{C}{E}$
    & \multicolumn{4}{c|}{Networkaddresses} \\  
    & 25 & 26 & 27
    & $\sum_{25}^{27}$ &  &
    & $+S$
    & $A_{d}$ 
    & $A_{s}$
    & $+(A_{d},A_{s})$
    & Seg.
  
    \\  
    \hline
    \hline
    FIN    & 5 & 5 & 3 & 13 & 1 & 1 & 14 & 14 & 3 & 17 & /27 [32]\\
    \hline
    HR     & 3 & 3 & 2 & 8 & 3 & 1 & 11 & 11 & 3 & 14 & /28 [16]\\
    \hline
    MNG     & 5 & 5 & 3 & 13 & 0 & 1 & 13 & 13 & 3 & 16 & /28 [16]\\
    \hline
    SD     & 10 & 10 & 5 & 25 & 0 & 2 & 50 & 50 & 3 & 53 & /26 [64]\\
    \hline
    PROD     & 0 & 0 & 0 & 0 & 5 & 1 & 5 &  5 & 3 & 8 & /29 [8]\\
    \hline
    WLAN   & 23 & 23 & 13 & 59 & - & 2 & 118 & 118 & 3 & 121 & /25 [128]\\
    \hline
    \multicolumn{10}{r}{$\sum$}
    & 219
    & 264 \\
  \end{tabular}
  \end{tiny}
\end{frame}

\begin{frame}

  \frametitle{LF09:10::Lösung-B:Initiales Netzdesign}
  \begin{center}
    \includegraphics[height=7cm]{\imgLf/ipv4-segmentation-task-start-nw-design.png}
  \end{center}
\end{frame}

\begin{frame}[fragile]  
  \frametitle{LF09:10::Lösung-B:Bereinigte Bedarfsanalyse}
  
  \vspace{1cm}
  \begin{tiny}
  \renewcommand{\arraystretch}{1.5}
  \begin{tabular}{|c|c|c|c|c|c|c||c||c|c|c|c|c|}
      
    & \multicolumn{4}{c|}{Employees} %2-4 
    & \multirow{2}{*}{$S$} %5-6
    & \multirow{2}{*}{$\frac{C}{E}$} %7
    & $\sum_{25}^{27}{}*\frac{C}{E}$
    & \multicolumn{5}{c|}{Networkaddresses} \\  
    & 25 & 26 & 27
    & $\sum_{25}^{27}$ &  &
    & $+S$
    & $A_{d}$ 
    & $A_{s}$
    & $A_{r}$
    & $+(A_{d},A_{s},A_{r})$
    & Seg.
    \\  
    \hline
    \hline
    FIN    & 5 & 5 & 3 & 13 & 1 & 1 & 14 & 14 & 3 & 0 & 17 & /27 [32]\\
    \hline
    HR     & 3 & 3 & 2 & 8 & 3 & 1 & 11 & 11 & 3 & 1 & 15 & /28 [16]\\
    \hline
    MNG     & 5 & 5 & 3 & 13 & 0 & 1 & 13 & 13 & 3 & 2 & 18 & /27 [32]\\
    \hline
    SD     & 10 & 10 & 5 & 25 & 0 & 2 & 50 & 50 & 3 & 2 & 55 & /26 [64]\\
    \hline
    PROD     & 0 & 0 & 0 & 0 & 5 & 1 & 5 &  5 & 3 & 1 & 9 & /28 [16]\\
    \hline
    WLAN   & 23 & 23 & 13 & 59 & 0 & \textbf{1,03}\footnote{\tiny{Größtes Restnetz: /26 [64] - 3 obligatorische Adressen = 61 / 59 = 1,03 pro Person ;-) }} & 61 & 61 & 3 & 0 & 64 & /26 [64]\\
    \hline
    \hline
    RN-1   & 0 & 0 & 0 & 0 & 0 & 0 & 0 & 0 & 3 & 5 & 8 & /29 [8]\\
    \hline
    RN-2   & 0 & 0 & 0 & 0 & 0 & 0 & 0 & 0 & 3 & 1 & 4 & /30 [4]\\
    \hline
    \multicolumn{11}{r}{$\sum$}
    & 190
    & 236 \\
  
  \end{tabular}
  \end{tiny}
\end{frame}

\begin{frame}[fragile]  
  \frametitle{LF09:10::Lösung-B:Numerische Adressinstantiierung}
    \begin{columns}[t]
      \column{0.5\textwidth}
    \begin{tiny}
      \begin{tabular}{c l c l}
      \texttt{192.168.1.0/26} & \texttt{\#64} & \multirow{4}{*}{\tcrotate{WLAN}} &  \texttt{NA: 192.168.1.0} \\
      & \texttt{[.0,} &  & \texttt{BC: 192.168.1.63} \\
      & \texttt{..., } &  & \texttt{NM: 255.255.255.192} \\
      & \texttt{.63]} &  & \texttt{R1: 192.168.1.1} \\
      \multicolumn{4}{l}{} \\
      \multicolumn{4}{l}{\rule{0.95\textwidth}{0.4pt} } \\
      \texttt{192.168.1.64/26} & \texttt{\#64} & \multirow{4}{*}{\tcrotate{SD}} &  \texttt{NA: 192.168.1.64} \\
      & \texttt{[.64,} &  & \texttt{BC: 192.168.1.127} \\
      & \texttt{..., } &  & \texttt{NM: 255.255.255.192} \\
      & \texttt{.127]} &  & \texttt{GW: 192.168.1.65} \\
      &  &  & \texttt{R2: 192.168.1.66} \\
      &  &  & \texttt{R3: 192.168.1.67} \\
      \multicolumn{4}{l}{\rule{0.95\textwidth}{0.4pt} } \\
     \texttt{192.168.1.128/27} & \texttt{\#32} & \multirow{4}{*}{\tcrotate{FIN}} &  \texttt{NA: 192.168.1.128} \\
      & \texttt{[.128,} &  & \texttt{BC: 192.168.1.159} \\
      & \texttt{..., } &  & \texttt{NM: 255.255.255.224} \\
      & \texttt{.159]} &  & \texttt{GW: 192.168.1.129} \\
      \multicolumn{4}{l}{\rule{0.95\textwidth}{0.4pt} } \\
     \texttt{192.168.1.160/27} & \texttt{\#32} & \multirow{4}{*}{\tcrotate{MNG}} &  \texttt{NA: 192.168.1.160} \\
      & \texttt{[.160,} &  & \texttt{BC: 192.168.1.191} \\
      & \texttt{..., } &  & \texttt{NM: 255.255.255.224} \\
      & \texttt{.191]} &  & \texttt{GW: 192.168.1.161} \\
      &  &  & \texttt{R2: 192.168.1.162} \\
      &  &  & \texttt{R3: 192.168.1.163} \\
      \multicolumn{4}{l}{\rule{0.95\textwidth}{0.4pt} } \\
      \texttt{192.168.1.208/28} & \texttt{\#16} & \multirow{4}{*}{\tcrotate{PROD}} &  \texttt{NA: 192.168.1.208} \\
      & \texttt{[.208,} &  & \texttt{BC: 192.168.1.223} \\
      & \texttt{..., } &  & \texttt{NM: 255.255.255.248} \\
      & \texttt{.223]} &  & \texttt{GW: 192.168.1.209} \\
      & &  & \texttt{R1: 192.168.1.210} \\
      \end{tabular}
    \end{tiny}
    \column{0.5\textwidth}
    \begin{tiny}
      \begin{tabular}{c l c l}
      \texttt{192.168.1.192/28} & \texttt{\#16} & \multirow{4}{*}{\tcrotate{HR}} &  \texttt{NA: 192.168.1.192} \\
      & \texttt{[.192,} &  & \texttt{BC: 192.168.1.207} \\
      & \texttt{..., } &  & \texttt{NM: 255.255.255.240} \\
      & \texttt{.207]} &  & \texttt{GW: 192.168.1.193} \\
      &  &  & \texttt{R1: 192.168.1.194} \\
      \multicolumn{4}{l}{\rule{0.95\textwidth}{0.4pt} } \\
      \texttt{192.168.1.232/30} & \texttt{\#4} & \multirow{4}{*}{\tcrotate{RN1}} &  \texttt{NA: 192.168.1.232} \\
      & \texttt{[.232,} &  & \texttt{BC: 192.168.1.235} \\
      & \texttt{..., } &  & \texttt{NM: 255.255.255.252} \\
      & \texttt{.235]} &  & \texttt{GW: 192.168.1.233} \\   
      & &  & \texttt{R2: 192.168.1.234} \\   
      \multicolumn{4}{l}{} \\
      \multicolumn{4}{l}{\rule{0.95\textwidth}{0.4pt} } \\
      \texttt{192.168.1.224/29} & \texttt{\#8} & \multirow{4}{*}{\tcrotate{RN2}} &  \texttt{NA: 192.168.1.224} \\
      & \texttt{[.224,} &  & \texttt{BC: 192.168.1.231} \\
      & \texttt{..., } &  & \texttt{NM: 255.255.255.248} \\
      & \texttt{.231]} &  & \texttt{GW: 192.168.1.225} \\   
      & &  & \texttt{R2: 192.168.1.226} \\   
      & &  & \texttt{R3: 192.168.1.227} \\   
      & &  & \texttt{R4: 192.168.1.228} \\   
      & &  & \texttt{R5: 192.168.1.229} \\   
      & &  & \texttt{R6: 192.168.1.230} \\   
      \end{tabular}
    \end{tiny}
  \end{columns} 
\end{frame}

\begin{frame}

  \frametitle{LF09:10::Lösung-B:Finales Netzdesign}
  \begin{center}
  \includegraphics[height=7cm]{\imgLf/ipv4-segmentation-task-final-nw-design.png}
  \end{center}
\end{frame}

