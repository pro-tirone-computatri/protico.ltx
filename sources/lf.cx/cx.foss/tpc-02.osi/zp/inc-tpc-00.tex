% This file is part of the Open Source project 'proTironeComputatri'
% (c) 2025 Karsten Reincke (https://github.com/pro-tirone-computatri/protico.ltx)
% It is distributed under the terms of the creative commons license
% CC-BY-4.0 (= https://creativecommons.org/licenses/by/4.0/)

\selectlanguage{ngerman}

\begin{frame}  
  \frametitle{LFCX:FOSS:Begriffsgeschichte}
  \begin{center}
  \includegraphics[height=8cm]{\imgLf/crx-osi-cloud.png}
  \end{center}
\end{frame}

\begin{frame}  
  \frametitle{LFCX:FOSS:Die Vier Freiheiten der FSF}

  \begin{footnotesize}
    \begin{columns}[t]
    \column{0.65\textwidth}
    FSF kompatible Lizenzen müssen die Freiheit gewähren,

    \vspace{0.35cm}
    \begin{itemize}
      \item[1.] \enquote{\ [...] das Programm auszuführen wie man möchte, für jeden Zweck} .
      \item[2.] \enquote{\ [...] die Funktionsweise des Programms zu untersuchen und [...] anzupassen}.
      \item[3.] \enquote{\ [...] das Programm zu redistribuieren [...]}.
      \item[4.] \enquote{\ [...] das Programm zu verbessern und diese Verbesserungen der Öffentlichkeit freizugeben [...]}.
    \end{itemize}


    \column{0.35\textwidth}
      \begin{center}
      \includegraphics[height=2cm]{\imgGl/logo-fsf.png}
      \end{center}
    \end{columns}
    
    \vspace{0.35cm}
    \begin{tiny} 
    \begin{itemize}
      \item[vgl.] \href{https://www.gnu.org/philosophy/free-sw.de.html}{https://www.gnu.org/philosophy/free-sw.de.html}
      \item[vgl.] \href{https://www.gnu.org/licenses/license-list.de.html}{https://www.gnu.org/licenses/license-list.de.html}
    \end{itemize} 
    \end{tiny}
  \end{footnotesize}
\end{frame}

\begin{frame}  
  \frametitle{LFCX:FOSS:OSI und die 10 Kriterien}

  \begin{footnotesize}

    Gemäß Open-Source-Definition müssen Open-Source-Lizenzen

    \begin{columns}    
    \column{0.65\textwidth}


    \vspace{0.2cm}
    \begin{itemize}
      \item[1.] die freie Weitergabe garantieren.
      \item[2.] den Zugriff auf den Quellcode ermöglichen.
      \item[3.] abgeleitete Arbeiten (Veränderungen) erlauben.
      \item[4.] die Weitergabe abgeleiteter Arbeiten ermöglichen.
      \item[5.] Diskriminierungen von Personen oder Gruppen ausschließen.
      \item[6.] Nutzungseinschränkung verhindern.
      \item[7.] ihre Rechte allen Personen gewährleisten.
      \item[8.] produktspezifische Begrenzungen ausschließen.
      \item[9.] Ausgrenzungen anderer Lizenzen verhindern.
      \item[10.] technologieneutral sein.
    \end{itemize}


    \column{0.2\textwidth}
      \begin{center}
      \includegraphics[height=2cm]{\imgGl/logo-osi.png}
      \end{center}
    \end{columns}
    
    \vspace{0.35cm}
    \begin{tiny} 
    \begin{itemize}
      \item[vgl.] \href{https://de.wikipedia.org/wiki/Open\_Source\_Initiative}{https://de.wikipedia.org/wiki/Open\_Source\_Initiative}
      \item[vgl.] \href{https://opensource.org/osd}{https://opensource.org/osd}
    \end{itemize} 
    \end{tiny}
  \end{footnotesize}
\end{frame}


\begin{frame}  
  \frametitle{LFCX:FOSS:7 Open Source Mythen}
  \begin{center}
  \includegraphics[height=7cm]{\imgLf/crx-foss-myths.png}
  \end{center}
\end{frame}