% This file is part of the Open Source project 'proTironeComputatri'
% (c) 2025 Karsten Reincke (https://github.com/pro-tirone-computatri/protico.ltx)
% It is distributed under the terms of the creative commons license
% CC-BY-4.0 (= https://creativecommons.org/licenses/by/4.0/)

\selectlanguage{ngerman}

\begin{frame}[t]  
  \frametitle{LFCX:FOSS:Ein Katjes-Salat als Beispiel}
  
  \begin{columns}[t]

    \column{0.46\textwidth}
    \begin{center}
      \includegraphics[height=3cm]{\imgLf/ktj-salad.png}
    \end{center}
    \begin{footnotesize}
      \begin{enumerate}
        \item Süße Katjes in die Schüssel.
        \item Gut vermischen.
        \item Saure Katjes obendrauf.
        \item Wohl bekomm's beim Verkosten.
      \end{enumerate}
    \end{footnotesize}
    \column{0.54\textwidth}
    \begin{footnotesize}

    \vspace{1.7cm}

    Ein veganer \textbf{Katjes-Salat} mit Rezept:
    
    \vspace{0.4cm}
    \textit{\textrm{Wem gehört das Rezept?}}
    \end{footnotesize}
  \end{columns}

\end{frame}

\begin{frame}[t]  
  \frametitle{LFCX:FOSS:Copyright-Owner}
  
  \begin{columns}[t]

    \column{0.46\textwidth}
    \begin{center}
      \includegraphics[height=3cm]{\imgLf/ktj-salad.png}
    \end{center}
    \begin{footnotesize}
      \begin{enumerate}
        \item Süße Katjes in die Schüssel.
        \item Gut vermischen.
        \item Saure Katjes obendrauf.
        \item Wohl bekomm's beim Verkosten.
      \end{enumerate}
    \end{footnotesize}
    \column{0.54\textwidth}
    \begin{footnotesize}

    \vspace{1.7cm}
    Ein veganer \textbf{Katjes-Salat} mit Rezept:
      
    \vspace{0.4cm}
    \begin{itemize}
      \item \textbf{\texttt{\copyright\ 2016 Karsten Reincke,\\ Deutsche Telekom AG}}
    \end{itemize}
    \end{footnotesize}
  \end{columns}

\end{frame}

\begin{frame}[t]  
  \frametitle{LFCX:FOSS:Nutzungsrechte}
  
  \begin{columns}[t]

    \column{0.46\textwidth}
    \begin{center}
      \includegraphics[height=3cm]{\imgLf/ktj-salad.png}
    \end{center}
    \begin{footnotesize}
      \begin{enumerate}
        \item Süße Katjes in die Schüssel.
        \item Gut vermischen.
        \item Saure Katjes obendrauf.
        \item Wohl bekomm's beim Verkosten.
      \end{enumerate}
    \end{footnotesize}
    %\column{0.1\textwidth}
    \column{0.54\textwidth}
    \begin{footnotesize}

    \vspace{1.7cm}
    Ein veganer \textbf{Katjes-Salat} mit Rezept:
      
    \vspace{0.4cm}
    \begin{itemize}
      \item \textbf{\texttt{\copyright\ 2016 Karsten Reincke,\\ Deutsche Telekom AG}}
    \end{itemize}
    
    \vspace{1.1cm}
      \textit{\textrm{Was dürfen Sie mit dem \underline{Salat} tun?}}\\
    \textit{\textrm{Was dürfen Sie mit dem Salat\underline{rezept} tun?}}
    \end{footnotesize}
  \end{columns}
\end{frame}

\begin{frame}[t]  
  \frametitle{LFCX:FOSS:Rechtevorbehalt}
  
  \begin{columns}[t]

    \column{0.46\textwidth}
    \begin{center}
      \includegraphics[height=3cm]{\imgLf/ktj-salad.png}
    \end{center}
    \begin{footnotesize}
      \begin{enumerate}
        \item Süße Katjes in die Schüssel.
        \item Gut vermischen.
        \item Saure Katjes obendrauf.
        \item Wohl bekomm's beim Verkosten.
      \end{enumerate}
    \end{footnotesize}
    \column{0.54\textwidth}
    \begin{footnotesize}

    \vspace{1.7cm}
    Ein veganer \textbf{Katjes-Salat} mit Rezept:
      
    \vspace{0.4cm}
    \begin{itemize}
      \item \textbf{\texttt{\copyright\ 2016 Karsten Reincke,\\ Deutsche Telekom AG}}
      \item \textbf{Alle Rechte vorbehalten.}
      \item \textit{Nachdruck oder Vervielfältigung - auch auszugsweise - nur mit ausdrücklicher, schriftlicher Genehmigung zulässig. }
    \end{itemize}
    \end{footnotesize}
  \end{columns}
\end{frame}

\begin{frame}[t]  
  \frametitle{LFCX:FOSS:Ein freier Katjes-Salat}
  
  \begin{columns}[t]

    \column{0.46\textwidth}
    \begin{center}
      \includegraphics[height=3cm]{\imgLf/ktj-salad.png}
    \end{center}
    \begin{footnotesize}
      \begin{enumerate}
        \item Süße Katjes in die Schüssel.
        \item Gut vermischen.
        \item Saure Katjes obendrauf.
        \item Wohl bekomm's beim Verkosten.
      \end{enumerate}
    \end{footnotesize}
    \column{0.54\textwidth}
    \begin{footnotesize}

    \vspace{1.7cm}

    Ein veganer \textbf{Katjes-Salat} mit Rezept:
    
    \vspace{0.4cm}
    \textit{\textrm{Und was dürften Sie, wenn das ein Open-Source-Rezept wäre?}}
    \end{footnotesize}
    \begin{center}
      %\hspace{0.5cm}
      \includegraphics[height=2cm]{\imgGl/logo-osi.png}
    \end{center}
  \end{columns}

\end{frame}

\begin{frame}[t]  
  \frametitle{LFCX:FOSS:Ein freier Katjes-Salat}
  
  \begin{columns}[t]

    \column{0.46\textwidth}
    \begin{center}
      \includegraphics[height=3cm]{\imgLf/ktj-salad.png}
    \end{center}
    \begin{footnotesize}
      \begin{enumerate}
        \item Süße Katjes in die Schüssel.
        \item Gut vermischen.
        \item Saure Katjes obendrauf.
        \item Wohl bekomm's beim Verkosten.
      \end{enumerate}
    \end{footnotesize}
    \column{0.54\textwidth}
    \begin{footnotesize}

    \vspace{1.7cm}

    Einen freien \textbf{Katjes-Salat} mit Rezept dürften Sie
    
    \vspace{0.4cm}
    \begin{enumerate}
    \item \textbf{verstehen} (= \textit{dafür das Rezept})
    \item \textbf{verwenden} (= \textit{anrichten + essen})
    \item \textbf{verbessern} (= \textit{pimpen})
    \item \textbf{verbreiten} (= \textit{anderen geben})
    \end{enumerate}
    
    \begin{tiny}
    \begin{itemize} 
      \item[=] FSFE: \href{https://fsfe.org/freesoftware/freesoftware.de.html}{Die vier Freiheiten freier 'Software'}
      \item[=] FSF: \href{https://www.gnu.org/philosophy/free-sw.de.html}{Die vier wesentliche Freiheiten}
    \end{itemize}
    \end{tiny}

    \textit{\textrm{Und wie organisiert man das?}}
    \end{footnotesize}
  \end{columns}

\end{frame}

\begin{frame}[t]  
  \frametitle{LFCX:FOSS:Lizenzierung}
  
  \begin{columns}[t]

    \column{0.46\textwidth}
    \begin{center}
      \includegraphics[height=3cm]{\imgLf/ktj-salad.png}
    \end{center}
    \begin{footnotesize}
      \begin{enumerate}
        \item Süße Katjes in die Schüssel.
        \item Gut vermischen.
        \item Saure Katjes obendrauf.
        \item Wohl bekomm's beim Verkosten.
      \end{enumerate}
    \end{footnotesize}
    \column{0.54\textwidth}
    \begin{footnotesize}

    \vspace{1.7cm}
    Einen freien \textbf{Katjes-Salat} mit Rezept
    
    \vspace{0.4cm}
    hätte der Urheber/Rechteinhaber erweitert, um 
    \vspace{0.4cm}
      
    \begin{itemize}
      \item die \textbf{Copyrightline}, z.B. \\ \begin{tiny}{\textit{\texttt{"\copyright\ 2016 K. Reincke, Deutsche Telekom AG"}}}\end{tiny}
      \item ein \textbf{Lizenzierungstatement}, z.B. \\ \begin{tiny}{\textit{"veröffentlicht unter den Bedingungen der MIT-Lizenz"}}\end{tiny}
    \end{itemize}
    \end{footnotesize}
  \end{columns}

\end{frame}

\begin{frame}[t]  
  \frametitle{LFCX:FOSS:Lizenzoptionen}
  
  \begin{columns}[t]

    \column{0.35\textwidth}
    \underline{\textit{Ohne Verpflichtungen:}}

    \vspace{0.5cm}
    \textbf{Public Domain}

    \vspace{0.2cm}
    \includegraphics[height=1cm]{\imgGl/logo-pd.png}

    \column{0.65\textwidth}
    \underline{\textit{Mit Verpflichtungen (= \textbf{Paying-By-Doing}):}}
    
    \vspace{0.5cm}
    \textbf{permissive} \textit{Open-Source-Lizenzen}

    \vspace{0.2cm}
    \includegraphics[height=1cm]{\imgGl/logo-bsd.png}
    \includegraphics[height=1cm]{\imgGl/logo-mit.png} 
    \includegraphics[height=1cm]{\imgGl/logo-apache.png}
    ...

    \vspace{0.5cm}
    \textit{Open-Source-Lizenzen} mit \textbf{schwachem Copyleft} 

    \vspace{0.2cm}
    \includegraphics[height=1cm]{\imgGl/logo-lgpl.png}
    \includegraphics[height=1cm]{\imgGl/logo-mozilla.png}
    \includegraphics[height=1cm]{\imgGl/logo-eclipse.png}
    ...

    \vspace{0.5cm}
    \textit{Open-Source-Lizenzen} mit \textbf{starkem Copyleft} 

    \vspace{0.2cm}
    \includegraphics[height=1cm]{\imgGl/logo-gpl.png} \includegraphics[height=1cm]{\imgGl/logo-agpl.png}
    ...
    
  \end{columns}

\end{frame}

\begin{frame}[t]  
  \frametitle{LFCX:FOSS:Paying By Doing}
   
  \begin{columns}[t]

    \column{0.46\textwidth}
    \begin{center}
      \includegraphics[height=3cm]{\imgLf/ktj-salad.png}
    \end{center}
    \begin{footnotesize}
      \begin{enumerate}
        \item Süße Katjes in die Schüssel.
        \item Gut vermischen.
        \item Saure Katjes obendrauf.
        \item Wohl bekomm's beim Verkosten.
      \end{enumerate}
    \end{footnotesize}
    \column{0.54\textwidth}
    \begin{footnotesize}

    %\vspace{1.7cm}

    Ein freies \textbf{Katjes-Salat}-Rezept - mit Kosten? 

    \vspace{0.2cm}
    Open-Source, die \textbf{nicht kostenlos} ist? 
    
    \vspace{0.2cm}
    \textit{Die Nutzungsrechte 'erwerben' nach dem Prinzip \textbf{Paying-By-Doing?}}
    
    \begin{center}
      %\hspace{0.5cm}
      \includegraphics[height=2cm]{\imgGl/logo-osi.png}
    \end{center}
    
    \vspace{0.3cm}
    \textbf{Genau!} Für das Umsetzen des Tuns, das die Lizenzen einfordern, muss Geld eingeplant und ausgegeben werden.


    \vspace{0.3cm}
    Und das heißt konkret \textbf{was?} 
    \end{footnotesize}
  \end{columns}

\end{frame}

\begin{frame}[t]  
  \frametitle{LFCX:FOSS:Paying By Doing}

  \begin{footnotesize}
  
    \begin{center}Von der Open-Source-Distributorin fordern...\end{center} 

  \begin{columns}[t]

    \column{0.46\textwidth}
    \begin{center}
    ... alle Lizenzen \textbf{- auch die permissiven}:

    \vspace{0.4cm}

      \begin{tabular}{c c}
        \includegraphics[height=2cm]{\imgGl/shouting-349672-oca.png}
        &
        \includegraphics[height=2cm]{\imgGl/license-341956-oca.png}
        \\
        & \\
        \includegraphics[height=2cm]{\imgGl/cr-227032-oca.png} 
        &
        \includegraphics[height=2cm]{\imgGl/attention-348409-oca.png} 
        \\
      \end{tabular}

    \end{center}

    \column{0.08\textwidth}

    \column{0.46\textwidth}
    \begin{center}
      \textbf{Also hier}:

      \vspace{0.5cm}
      KEINE Pflicht, immer auch das Salatrezept (den Code) mitzugeben.

      \vspace{1cm}
      \textbf{und}:
      
      \vspace{0.5cm}
      KEINE Pflicht, immer auch die Rezepte des ganzen Menüs (den komplexen Code) mitzugeben, in dem der Salat (die Komponente) verwendet wird.


    \end{center}  


  \end{columns}

  \end{footnotesize}
\end{frame}

\begin{frame}[t]  
  \frametitle{LFCX:FOSS:Paying By Doing}

  \begin{footnotesize}
  
    \begin{center}Von der Open-Source-Distributorin fordern...\end{center} 

  \begin{columns}[t]

    \column{0.46\textwidth}
    \begin{center}
    ... \textbf{alle Lizenzen} \textit{- die permissiven}:

    \vspace{0.4cm}

      \begin{tabular}{c c}
        \includegraphics[height=2cm]{\imgGl/shouting-349672-oca.png}
        &
        \includegraphics[height=2cm]{\imgGl/license-341956-oca.png}
        \\
        & \\
        \includegraphics[height=2cm]{\imgGl/cr-227032-oca.png} 
        &
        \includegraphics[height=2cm]{\imgGl/attention-348409-oca.png} 
        \\
      \end{tabular}

    \end{center}

    \column{0.08\textwidth}

    \column{0.46\textwidth}
    \begin{center}
      ... und die mit \textbf{schwachem Copyleft}:

      \vspace{0.2cm}
      \begin{tabular}{ m{2cm} m{2cm}}
        \huge{+}
        &
         \includegraphics[height=2cm]{\imgGl/codefile-000-oca.png}
        \\
      \end{tabular}

      \vspace{0.4cm}

      \textbf{Also auch} bei \textit{schwachem Copyleft}
      
      \vspace{0.2cm}
      KEINE Pflicht, immer auch die Rezepte des ganzen Menüs (den komplexen Code) mitzugeben, in dem der Salat (die Komponente) verwendet wird.

    \end{center}  


  \end{columns}

  \end{footnotesize}
\end{frame}

\begin{frame}[t]  
  \frametitle{LFCX:FOSS:Paying By Doing}

  \begin{footnotesize}
  
    \begin{center}Von der Open-Source-Distributorin fordern...\end{center} 

  \begin{columns}[t]

    \column{0.46\textwidth}
    \begin{center}
    ... \textbf{alle Lizenzen} \textit{- die permissiven}:

    \vspace{0.4cm}

      \begin{tabular}{c c}
        \includegraphics[height=2cm]{\imgGl/shouting-349672-oca.png}
        &
        \includegraphics[height=2cm]{\imgGl/license-341956-oca.png}
        \\
        & \\
        \includegraphics[height=2cm]{\imgGl/cr-227032-oca.png} 
        &
        \includegraphics[height=2cm]{\imgGl/attention-348409-oca.png} 
        \\
      \end{tabular}

    \end{center}

    \column{0.08\textwidth}

    \column{0.46\textwidth}
    \begin{center}
      ... die mit \textbf{schwachem Copyleft}:

      \vspace{0.2cm}
      \begin{tabular}{ m{2cm} m{2cm}}
        \huge{+}
        &
         \includegraphics[height=2cm]{\imgGl/codefile-000-oca.png}
        \\
      \end{tabular}

      \vspace{0.4cm}
      ... und die Lizenzen mit \textbf{starkem Copyleft}:
      
      \vspace{0.2cm}
      \begin{tabular}{ m{2cm} m{2cm}}
        \huge{+}
        &
         \includegraphics[height=2cm]{\imgGl/codefiles-000-oca.png}
        \\
      \end{tabular}

    \end{center}  


  \end{columns}

  \end{footnotesize}
\end{frame}

\begin{frame}[t]  
  \frametitle{LFCX:FOSS:Summary}

  \begin{footnotesize}
  \begin{columns}[t]

    \column{0.46\textwidth}
    \textbf{Opensourcelizenzen}:

    
      \begin{itemize}
        \item sprechen uns Nutzungsrechte zu!
        \item verlangen eine Gegenleistung von uns!
      \end{itemize}
    

    \column{0.08\textwidth}
    \column{0.46\textwidth}
    \textbf{Opensourcelizenzierungen}:

    
      \begin{itemize}
        \item beziehen sich auf eine Open-Source-Lizenz!
        \item basieren auf einem Copyrightstatement
        \item manifestieren sich in einem Lizenzierungsstatement!
      \end{itemize}
    

  \end{columns}
\end{footnotesize}

\begin{center}
  \includegraphics[height=4cm]{\imgLf/foss-license-system.png}
\end{center} 
\end{frame}

\begin{frame}[t]  
  \frametitle{LFCX:FOSS:Analogie}

    \begin{center}
      \includegraphics[height=6.5cm]{\imgGl/foss-salad-analogy.png}
    \end{center}


\end{frame}

\begin{frame}[t]  
  \frametitle{LFCX:FOSS:Derivative Work}
  \begin{footnotesize}
  \begin{center}
    \includegraphics[height=5cm]{\imgGl/foss-dv-prese.png}
  \end{center} 
    
  \vspace{0.4cm}
  \hfill \textit{\textrm{Wem gehört diese Präsentation?}}
  \end{footnotesize}

\end{frame}

\begin{frame}[t]  
  \frametitle{LFCX:FOSS:Derivative Work}
  \begin{footnotesize}

  \begin{center}
    \includegraphics[height=5cm]{\imgGl/foss-dv-prese.png}
  \end{center} 
    
  \vspace{0.4cm}
  \hfill \texttt{\copyright\ 2025 Karsten Reincke, Deutsche Telekom AG}

  Allerdings ...
  \end{footnotesize}

\end{frame}

