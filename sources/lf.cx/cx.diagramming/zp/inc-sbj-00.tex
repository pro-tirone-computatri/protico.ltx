% This file is part of the Open Source project 'proTironeComputatri'
% (c) 2025 Karsten Reincke (https://github.com/pro-tirone-computatri/protico.ltx)
% It is distributed under the terms of the creative commons license
% CC-BY-4.0 (= https://creativecommons.org/licenses/by/4.0/)

\selectlanguage{ngerman}

{
\usebackgroundtemplate{\includegraphics[width=\paperwidth]{\imgLf/chart-774947-pxh.png}}
\begin{frame}  
  \frametitle{LFCX:Diagramme:UML}

  \vspace{6cm}
  \begin{flushright}
    \textcolor{blue}{Crashkurs UML}
  \end{flushright}
\end{frame}
}

\begin{frame}[t]
  \frametitle{LFCX:Diagramme:UML:Systematik\footcite[nach][8]{OestSche2014a}}
  
  \begin{footnotesize}
  \begin{columns}
    \column{0.5\textwidth}
      \textbf{Strukturdiagramme}:
      \begin{itemize}
        \item Klassendiagramm 
        \item Objektdiagramm 
        \item Komponentendiagramm
        \item Kompositionsstrukturdiagramm
        \item Verteilungsdiagramm
        \item Paketdiagramm
        \item Profildiagramm 
        \item Anwendungsfalldiagramm 
      \end{itemize}

    \column{0.5\textwidth}
      \textbf{Verhaltensdiagramme}:
      \begin{itemize}
        \item Aktivitätsdiagramm 
        \item Interaktionsdiagramm
          \begin{itemize}
            \item Interaktionsübersicht
            \item Kommunikationsdiagramm
            \item Sequenzdiagramm 
            \item Zeitdiagramm
          \end{itemize}
        \item Zustandsdiagramm 
          \begin{itemize}
            \item Protokollautomat
          \end{itemize}
      \end{itemize}
  \end{columns}
  \end{footnotesize}
  \vspace{1cm}
  \tiny{Hinweis: Das Anwendungsfalldiagramm wird oft auch als Verhaltensdiagramm klassifiziert}
\end{frame}

\begin{frame}
  \frametitle{LFCX:Diagramme:UML:Kontext}
  
  \begin{footnotesize}
    \begin{columns}[t]
    
    \column{0.5\textwidth}
      \textbf{Structural Diagrams}:
      \begin{itemize}
        \item \textbf{Class Diagram} \textrm{\textit{(beschreibt Klassen von Objekten, ihre Eigenschaften, Operationen und Beziehungen untereinanderS)}}
        \item ...
        \item \textbf{Object Diagram} \textrm{\textit{(ist eine zur Laufzeit vorhandene Einheit und ein Exemplar einer Klasse)}}      
        \item \textbf{Use Case Diagram} \textrm{\textit{(zeigt Akteure, Anwendungsfälle und deren Beziehungen untereinander)}}
      \end{itemize}

    \column{0.5\textwidth}
    \textbf{Behavioral Diagrams}:
      \begin{itemize}
        \item \textbf{Activity Diagram} \textrm{\textit{(beschreibt einen Ablauf mittels Knoten und Kontrollflüssen)}}
        \item Interaktionsdiagramm
        \item ...
          \begin{itemize}
            \item \textbf{Sequence Diagram} \textrm{\textit{(zeigt eine Reihe von Nachrichten, die von den Beteiligten in einer bestimmten Reihenfolge ausgetauscht werden.)}}
          \end{itemize}
        \item \textbf{State Diagramm} \textrm{\textit{beschreibt den Zustand (Stand der jeweiligen Daten) der Komponenten eines Systems}}
        \item ...
      \end{itemize}
  \end{columns}
  \end{footnotesize}
\end{frame}

\begin{frame}
  \frametitle{LFCX::UML:Aktivitätsdiagramm:Symbole}
  \begin{center}
    \includegraphics[height=7.8cm]{\imgGl/uml-activity-elements.png}
  \end{center}
\end{frame}

\begin{frame}
  \frametitle{LFCX::UML:Aktivitätsdiagramm:Beispiel}
  \begin{center}
  \includegraphics[height=7.8cm]{\imgGl/uml-activity-example.png}
  \end{center}
\end{frame}

\begin{frame}
  \frametitle{LFCX::UML:Sequenzdiagramm:Symbole}
  \begin{center}
  \includegraphics[height=7.8cm]{\imgGl/uml-sequence-elements.png}
  \end{center}
\end{frame}

\begin{frame}
  \frametitle{LFCX::UML:Sequenzdiagramm:Beispiel}
  \begin{center}
  \includegraphics[height=7.8cm]{\imgGl/uml-sequence-example.png}
  \end{center}
\end{frame}

\begin{frame}
  \frametitle{LFCX::Flussdiagramm:Symbole}
  \begin{center}
  \includegraphics[width=\textwidth]{\imgGl/flowchart-symbols.png}
  \end{center}
\end{frame}

\begin{frame}
  \frametitle{LFCX::Flussdiagramm:Erweiterte Symbole}
  \begin{center}
  \includegraphics[width=\textwidth]{\imgGl/flowchart-symbols-supplement.png}
  \end{center}
\end{frame}

\begin{frame}
  \frametitle{LFCX::Flussdiagramm:Beispiel}
  \begin{center}
  \includegraphics[height=7.5cm]{\imgGl/flowchart-example.png}
  \end{center}
\end{frame}

\begin{frame}
  \frametitle{LFCX::\underline{B}usiness \underline{P}rocess \underline{M}odel + \underline{N}otation: Symbole}
  \begin{center}
  \includegraphics[height=7cm]{\imgGl/bpmn-elements.png}
  \end{center}
\end{frame}

\begin{frame}
  \frametitle{LFCX::\underline{B}usiness \underline{P}rocess \underline{M}odel + \underline{N}otation: Beispiel}
  \begin{center}
  \includegraphics[height=7.5cm]{\imgGl/bpmn-example.png}
  \end{center}
\end{frame}

\begin{frame}
  \frametitle{LFCX::\underline{E}reignisgesteuerte \underline{P}rozess-\underline{K}ette: Symbole}
  \begin{center}
  \includegraphics[width=\textwidth]{\imgGl/epc-elements.png}
  \end{center}
\end{frame}

\begin{frame}
  \frametitle{LFCX::\underline{E}reignisgesteuerte \underline{P}rozess-\underline{K}ette: Beispiel}
  \begin{center}
  \includegraphics[height=7.5cm]{\imgGl/epc-example.png}
  \end{center}
\end{frame}


\begin{frame}
  \frametitle{LFCX::Synopse der Prozessdiagrammsymbole}
  \begin{center}
  \includegraphics[height=7cm]{\imgGl/symbols-synopse.png}
  \end{center}
\end{frame}
